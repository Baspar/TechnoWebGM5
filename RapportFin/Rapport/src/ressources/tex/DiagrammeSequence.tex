\chapter{Diagrammes de sequence}
    Pour faciliter la compréhension de notre programme, nous avons réalisé les différents diagrammes de séquence suivants. 

    \section{Affichage Village}

    On affiche au joueur le plan de son village. Nous récupérons le joueurCourant en session. On affiche d'abord le plan avec les batiments déjà placés puis l'ensemble des batiments non placés. 
    
        \includegraphics[scale=0.7]{graph/DiagrammeSequenceAffichageVillage.png}
        
    
    \section{Amelioration Batiment}

    En fonction de la page sur laquelle se trouve le joueur (Caserne, Hotel de Ville, ...) il sera proposé d'amélioré les batiments correspondant. Une fois l'amélioration demandée par le joueur, la servlet réalise l'opération si cela est possible. L'amélioration est réalisée sur le joueur en session mais également dans la bdd.
        \includegraphics[scale=0.4]{graph/DiagrammeSequenceAmeliorationBatiment.png}
        
    
    \section{Combat}
        Lors de l'execution principale du combat, une boucle sera effectuée sur:
        \begin{itemize}
            \item Chacun des personnages (dont les points de vie sont supérieurs à zéro)
            \item Chacun des batiments (dont les points de vie sont supérieurs à zéro)
        \end{itemize}
        Pour chacun d'entre eux, si une attaque est possible, ils attaqueront un enemi (Voire plus) s'il existe, et pour les soldats, se deplaceront le cas echéant. \\
        \includegraphics[scale=0.5]{graph/DiagrammeSequenceCombat.png}    
    
    \section{Connexion}

    Lors de la connexion du joueur, la première étape est la saisie des données par l'utilisateur. Puis la servlet est lancée, elle demande la vérification du mot de passe au Formulaire de connexion. On vérifie grâce à Jasypt, bibliothéque java qui permet d'encrypter le mot de passe, que le mot de passe saisi par l'utilisateur correspond bien au mot de passe encrypté dans la base de donné. Puis on renvoie un joueur ou des erreurs dans la session. \\
    Pour l'inscription le principe est similaire. \\
        \includegraphics[scale=0.4]{graph/DiagrammeSequenceConnexion.png}  
        
    \section{Creation Batiment}
    Principe similaire a l'amelioration. On propose au joueur de créer les batiments possibles puis on met à jour le joueur en session et en base de données. \\
        \includegraphics[scale=0.4]{graph/DiagrammeSequenceCreationBatiment.png}  
        
   \section{Creation Unite}
   De même que précedemment pour la creation du batiment. \\

        \includegraphics[scale=0.4]{graph/DiagrammeSequenceCreationUnite.png}    
        
   \section{Deplacement}

    Pour le déplacement d'un batiment, on enregistre le batiment sur lequel joueur clique en session puis si la position ou il clique la deuxième est valide, on déplace le batiment en session et dans la bdd. \\ 
    
        \includegraphics[scale=0.4]{graph/DiagrammeSequenceDeplacement.png}    
        
 
