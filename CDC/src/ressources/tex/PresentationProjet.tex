\chapter{Présentation du projet}
	Le but de notre projet est de réaliser une version simplifiée de Clash Of Clan

    \section{Présentation de Clash of Clan}
		Le but du jeu est de construire son village et de le défendre face aux autres joueurs. Pour construire son village, le joueur doit se servir des ressources (or, elixir, elixir noir) qu'il peut produire grâce à des mines et stocker dans des réservoirs. \\
		Ces ressources lui permettent d'améliorer ses bâtiments défensifs (canon, mortier) mais aussi de pouvoir construire de nouveaux camps militaire (pour avoir plus de soldats) ou d'améliorer ses unités. \\
		Pour gagner des ressources, le joueur a la possibilité d'attaquer d'autres villages pour les piller. Il faut donc être stratégique pour les attaques mais aussi pour la gestion des ressources. \\
		Le joueur a également la possibilité de rejoindre un clan. Une fois le clan rejoins il peut participer à des guerres qui lui fournissent un butin supplémentaire.      
        
    \section{Hypothèses simplificatrices}
		Ce jeu étant bien trop compliqué pour être réalisé dans le temps imparti nous avons choisi d'effectuer quelques hypothèses simplificatrices : 
\begin{itemize}
\item Nous n'allons pas prendre en compte le fait qu'un joueur puisse rejoindre un clan. 
\item Nous avons décidé de simplifier la gestion de l'armée : la construction des unités ainsi que l'amélioration des unités se fera dans le même bâtiment
\item Il n'y aura ni réservoir, ni camp d’entraînement : la quantité maximale pour chaque ressources ainsi que le nombre maximum de soldat sera déterminé par le niveau de l'hôtel de ville du joueur
\end{itemize}