\chapter{Technologies et méthodologie employées}
    \section{Technologies}
        Pour ce projet, nous allons être amenés a travailler sur 3 points différents.
            \subsection{Modèle}
                Le choix de la technologie qui sera employée pour le modèle est primordiale, car elle impliquera la technologie utilisée dans l'interface Web.\\
                Dans un projet de ce type, on peut clairement identifier une forte composante orientée objet.\\
                Parmi tous les langages disponibles, nombreux sont ceux qui vont nous permettre de réaliser de l'objet, tout en étant interfaçable avec un langage/framework de développement. De manière non exhaustive:
                \begin{description}
                    \item[Java] couplé avec le framework \textbf{J2EE}
                    \item[Python] couplé avec le framework \textbf{Django}
                    \item[Ruby] couplé avec le framework \textbf{Ruby On Rails}
                \end{description}
                Java étant un langage que nous avons eu l'occasion de manipuler tous deux, Java/J2EE parait être le meilleur couple à notre disposition.
            \subsection{Interface Web}
                Comme dit précédemment, Suite au choix de Java comme langage de développement, nous choisirons J2EE comme framework de développement Web
            \subsection{Stockage}
                Pour le stockage des données, deux solutions évidentes viennent à l'esprit:
                \begin{itemize}
                    \item Stockage dans des fichiers
                    \item Stockage dans des Bases de Données
                \end{itemize}
                MySQL collaborant très conjointement avec J2EE, et étant bien plus sûr qu'une écriture en dur sur des fichiers, nous choisissons cette technologie.
    \section{Methodologies}
        Pour realiser ce projets, nous allons utiliser une méthodologie bien definie:\\
        \begin{itemize}
            \item Le développement sera assisté par \textbf{Eclipse}, aussi bien pour la rédaction que pour la compilation.
            \item Le partage de projet se fera par un dépot \textbf{Git} hébergé sur GitHub
            \item Pour le développement en Java, un script Bash tiendra à jour notre diagramme de classes, pour fournir a tout isntant un diagramme conforme a notre avancement (Grâce a des balises //TODO, //WIP et //DONE correspondant a l'avancée)
            \item Le rapport sera rédigé en \textbf{Latex}
            \item L'intégralité des diagramme seront réalisés en \textbf{PlantUML}
        \end{itemize}
